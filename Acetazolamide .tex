\documentclass[11pt, ]{article}
\title{An Unusual AKI}
\author{Robert K. Forsythe}
\usepackage{todo}
\usepackage{graphicx}
\usepackage{subcaption}
\graphicspath{ {./images/} }


\usepackage[backend=bibtex,style=numeric,autocite=superscript,natbib=true,sorting=none]{biblatex} 
\addbibresource{Refs.bib}

\begin{document}
\maketitle
\begin{abstract}
	Acute Kidney Injury is one of the most common reasons for hospital admission and are often a side effect of prescribed medications. Acetazolamide is a medication prescribed occasionally for diverse indications including idiopathic intracranial hypertension, glaucoma, prophylaxis of mountain sickness and ventilator alkalosis. Although it is a medication we prescribe relatively often it is not a benign drug and adverse effects include allergic reactions, acidosis and electrolyte derangement. It is a sulphonamide and like other drugs of this class it can cause impaired renal function by a variety of mechanisms including interstitial nephritis, tubular necrosis and crystalluria. Here we describe an unusual AKI caused by acetazolamide.
\end{abstract}

\section*{Introduction}

		
\section*{Case}

A 72 year old lady attended the emergency department complaining of right sided loin pain for the preceding 36 hours. She had a medical history of hyperparathyroidism, hypothyroidism and a rectocele repaired 6 years ago. Forty-eight hours prior to her attendance she had repair of a macular hole and had been prescribed acetazolamide 250mg twice daily. She had taken 5 doses of this before discontinuing the medication due to nausea, vomiting and loin pain. She then attended A\&E. Blood tests on presentation demonstrated reduced renal function with a creatinine of 222umol/L from a baseline of around 70umol/L. Initially she was assessed with a non-contrast CTKUB. This did not demonstrate any nephrolithiasis but did suggest some fullness of the collecting system. Given the CT findings she was further imaged with a renal tract ultrasound which excluded any hydronephrosis. 

Unfortunately her renal function continued to decline and a central line was placed and 4 days following admission she was dialysed. Her renal function then made a very rapid recovery and she entered a polyuric phase. 

\begin{figure}
\begin{subfigure}{.5\textwidth}
\includegraphics[width=\textwidth]{InputOutput}
\caption{Graph of Daily Fluid Balance}
\end{subfigure}
\begin{subfigure}{.5\textwidth}
\includegraphics[width=\textwidth]{ubc}
\caption{Graph of Daily Biochemistry}
\end{subfigure}
\end{figure}


\begin{figure}
\begin{subfigure}{.5\textwidth}
\includegraphics[width=\textwidth]{rtkidus}
\caption{Ultrasound of right kidney}
\end{subfigure}
\begin{subfigure}{.5\textwidth}
\includegraphics[width=\textwidth]{ltkidus}
\caption{Ultrasound of left kidney}
\end{subfigure}
\end{figure}


\begin{figure}
\begin{subfigure}{.5\textwidth}
\includegraphics[width=\textwidth]{rtkidct}
\caption{CT of right kidney}
\end{subfigure}
\begin{subfigure}{.5\textwidth}
\includegraphics[width=\textwidth]{ltkidct}
\caption{CT of left kidney}
\end{subfigure}
\end{figure}



\section*{Discussion}
A brief review of the literature demonstrated that similar presentations of acute kidney injury have previously been encountered in assocation with acetazolamide\cite{Neyra2014, Rossert1984, Lawson2020}. In each of these cases acetazolamide has been associated with renal colic and anuria. Lawson found that the AKI resolved with aggressive fluid therapy and Neyra's patient required 2 sessions of haemodialysis. Acetazolamide is recognised to be associated with metabolic drug-induced calcium phophate or oxalate calculi, however in this case these were excluded by CT imaging. An older case published by Rossert in 1989 describes a similar case in which a biopsy was performed and demonstrated Tamm-Horsfall protein within the glomeruli and tubular lesions associated with intratubular crystal formation.This suggests intratubular obstruction and retrograde flow of tubular urine\cite{Rossert1984}.  

\section*{Conclusion}
When prescribing medications it is important to consider the harms that these medications may cause. Here we describe an unusual but life-threatening side effect of acetazolamide. We must be mindful of recent changes to medications when considering possible aetiology of unusual presentations.


\printbibliography
\end{document}
